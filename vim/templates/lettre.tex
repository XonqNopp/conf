%%
%%% Headers %%% {{{
%%% Document variables %%% {{{
\newcommand*{\Gael}{Ga\"el #&%}
\newcommand*{\ThisAuthors}{\Gael}
\newcommand*{\ThisTitle}{#&%}
\newcommand*{\ThisTitleSHORT}{\ThisTitle}
%%% End of document variables %%% }}}
\documentclass[romand,12pt,a4paper]{lettre}
%%
%%% Usepackages %%% {{{
\usepackage{mailing}
\usepackage[french,english]{babel}
%%% End of usepackages %%% }}}
%%
%%
\setcounter{tocdepth}{2}
\hyphenation{}
%\addressfile{#&%.mail}
%%
%%
%% }}}
%%
\begin{document}
\setlength{\parskip}{0.5ex}
%%
\begin{letter}{#&% Addressee address}
%%
%% {{{ Here is me (Be careful: by using the package "mailing", you can only change these parameters in the file /usr/local/texlive/2007/texmf-dist/tex/latex/lettre/default.ins)
\name{\Gael}
\lieu{Belfaux}
\date{le \today}
\notelephone
%\telephone{026/534.64.01}
%\telephone{078/847.22.08}
\nofax
\address{%
	\Gael\\
	Champ Bonajrd 1\\%
	1782 Belfaux\\%
	#&%
}
\signature{\Gael}
\conc{\title}
\romand
%% }}}
%%
\opening{#&% Opening greetings}
%%
%\mailingtext{}
%%
#&%
%%
\closing{#&% Closing greetings}
%%
\end{letter}
%%
%\makemailing
%%
\end{document}
%%
