%% New commands %%% {{{
%% Environments {{{
%% Subfigure without main caption {{{
\newenvironment{SubfigWoMainCap}{%
\renewcommand{\thesubfigure}{\thefigure\alph{subfigure}}
\begin{figure}[H]%
\begin{center}%
}{%
\vspace{-5mm}%
\end{center}%
\stepcounter{figure}
\vspace{-5mm}%
\end{figure}%
\renewcommand{\thesubfigure}{\alph{subfigure}}
}
%% }}}
%% }}}
%% References %% {{{
\newcommand*{\FigRef}[1]{\ref{fig:#1}}
\newcommand*{\TabRef}[1]{\ref{tab:#1}}
\newcommand*{\RefP}[1]{\S\ref{#1} p.~\pageref{#1}}
\newcommand*{\RefPP}[1]{\S\ref{#1} (p.~\pageref{#1})}
%% }}}
%% Hyperref {{{
\newcommand*{\mailto}[1]{\href{mailto:#1}{#1}}
\newcommand*{\biburl}[2]{\href{#1}{\texttt{#2} }}
%% \url{link}
%% \href{link}{text}
%% \href[ref]{text}
%% Form commands {{{
%% Must be placed in a 'Form' environment
%% \MyField{fieldname (like HTML form)}{additional options}{Text to be displayed before}
%% options: (double) width, (int) maxlen, (bool) multiline
%\newcommand{\MyField}[3]{\TextField[name=#1,backgroundcolor=1 1 0.95,bordercolor=0 0 0,bordersep=3,borderwidth=1,charsize=12pt,color=0 0 0,#2]{#3}}
%% \MyRadio{fieldname (like HTML form)}{wth}{wtf} %% prints a radio box AFTER text
%\newcommand{\MyRadio}[3]{\ChoiceMenu[name=#1,backgroundcolor=1 1 0.95,bordercolor=0 0 0,bordersep=3,borderwidth=1,charsize=12pt,color=0 0 0,radio=true,radiosymbol=\ding{54}]{#2}{#3}}
%% \MySingleRadio{fieldname (like HTML form)} %% only prints a radio box
%\newcommand{\MySingleRadio}[1]{\MyRadio{#1}{ }{ }}
%% }}}
%% }}}
%% Math %% {{{
%%\renewcommand{\vec}[1]{\ensuremath{\underline{#1}} }
%\renewcommand{\vec}[1]{\ensuremath{\boldsymbol #1}}
%\newcommand*{\mat}[1]{\ensuremath{\underline{\underline{#1}} }}
%%\newcommand*{\Rey}{\ensuremath{\mathcal{R}e}}
%\newcommand*{\dd}{\ensuremath{\textrm{d}} }
%\newcommand*{\Intdd}[1]{\ensuremath{\int \dd #1\ }}
%\newcommand*{\IntddLim}[2]{\ensuremath{\int\limits_{#2}\dd #1 }}
%\newcommand*{\IntddLims}[3]{\ensuremath{\int\limits_{#2}^{#3}\dd #1\ }}
%\newcommand*{\Avg}[1]{\ensuremath{\left\langle #1 \right\rangle}}
%%\newcommand*{\Grad}{\ensuremath{\vec{\nabla}} }
%\newcommand*{\Grad}{\ensuremath{\nabla}}
%\newcommand*{\Div}{\ensuremath{\Grad \cdot}}
%\newcommand*{\Curl}{\ensuremath{\Grad \wedge}}
%\newcommand*{\Lap}{\ensuremath{\nabla^2}}
%\newcommand*{\CrossP}[2]{\ensuremath{#1 \wedge #2}}
%\newcommand*{\CrossPV}[2]{\ensuremath{\vec{#1} \wedge \vec{#2}} }
%\newcommand*{\dsd}[2]{\ensuremath{\partial_{#2} #1}}
%\newcommand*{\dsdf}[2]{\ensuremath{\displaystyle\frac{\partial #1}{\partial #2}} }
%\newcommand*{\ddsd}[2]{\ensuremath{\dd_{#2} #1}}
%%\newcommand*{\ddsdf}[2]{\ensuremath{\displaystyle\frac{\dd #1}{\dd #2}} }
%\newcommand*{\dsdt}[1]{\ensuremath{\dsd{#1}{t}} }
%\newcommand*{\dsdtf}[1]{\ensuremath{\dsdf{#1}{t}} }
%\newcommand*{\dsdtfs}[1]{\ensuremath{\frac{\partial #1}{\partial t}} }
%%\newcommand*{\ddsdthf}[1]{\ensuremath{\displayfrac{\textrm{D} {#1}}{ extrm{D} t}} }
%%\newcommand*{\dhexpl}[2]{\ensuremath{\dsdt{#1} + ( #2 \cdot \grad ) #1}}
%% }}}
%% BibTeX %% {{{
%\newcommand{\FootnoteOne}[2]{\footnote{#2}\newcounter{#1}\setcounter{#1}{\value{footnote}} }
%\newcommand{\FootnoteRecall}[1]{\footnotemark[\value{#1}]}
%% }}}
%% Miscellaneous %% {{{
%\renewcommand\STprintnum[1]{\numprint{#1}}
%\npdecimalsign{.}
%\newcommand*{\Cpp}{C{\scriptsize ++}\ }
%\newcommand*{\Hpp}{C{\normalsize ++}\ }
\renewcommand*{\epsilon}{\varepsilon}
\renewcommand*{\roman}{\Roman}
%\renewcommand*{\geq}{\geqslant}
%\renewcommand*{\leq}{\leqslant}
\newcommand*{\oC}{\ensuremath{^{\circ}C}}
\AtBeginDocument{\renewcommand{\labelitemi}{\textbullet}}% Default is \textendash % must be placed after begin-doc
\makeatletter
\newcommand{\rom}[1]{\expandafter\@slowromancap\romannumeral #1@}
%\newcommand{\InlineTOC}{\@starttoc{toc}}%% For report inline toc (to use with the 'notitlepage' class option
\makeatother
%% }}}
%% Specific to document {{{
%% }}}
%% End of new commands %%% }}}
